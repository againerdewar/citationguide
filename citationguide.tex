\documentclass{article}
\usepackage{cctexexample}

\title{Citation guide}
\author{Carleton College \LaTeX{} workshop}
\date{}

\newcommand*{\code}[1]{\texttt{#1}}
\newcommand*{\filename}[1]{\texttt{#1}}
\newcommand*{\inst}[1]{\textbf{\boldmath{}#1}}

% Load in biblatex
% To use a different bibliography style, just change "numeric" to
% your preferred style (mla for MLA style, alphabetic for Author-Year
% style, etc.) There are a lot of options; check the BibLaTeX documentation.
\usepackage[backend=bibtex,style=numeric]{biblatex}
% Select the bibliography file
\addbibresource{sources.bib}

\begin{document}
\maketitle

\LaTeX{} provides powerful facilities for handling bibliographic citations in your documents through the Bib\TeX{} and Bib\LaTeX{} packages.
Like many other \LaTeX{} features, these will allow you to spend less time worrying about formatting and more time producing content.

Let's get started!

\section{What's Bib\LaTeX{}?}
\label{s:whatis}
Bib\LaTeX{} is a \enquote{citation management system}.
It allows you to handle citations automatically, the same way \LaTeX{} lets you handle internal references automatically with \code{\textbackslash{}label\{\}} and \code{\textbackslash{}cite\{\}}.
For example, in this document, I can cite the Bib\LaTeX{} documentation simply by typing \code{\textbackslash{}cite\{biblatex\}}; \LaTeX{} then prints the reference \cite{biblatex}, and the document is added to the list of references below.

\emph{This is awesome.}
Here's all the things you \emph{don't} have to do:
\begin{itemize}
\item
  You don't have to remember how to format the in-line citation.

\item
  You don't have to think about how to format the bibliography entry.

\item
  You don't have to keep track of what sources you cite in your document.

\item
  You don't have to sort the bibliography.
\end{itemize}

To get access to all this power, you need to know just a bit about how Bib\LaTeX{} works.

\section{The Bib\LaTeX{} database}
\label{s:database}
The core of the Bib\LaTeX{} system is the \emph{database}, which stores data about the documents you might want to cite.
It's a simple text file (usually with the \code{.bib} extension) that lives in the directory with your documents.
The database used by this document appears in \cref{lst:thedatabase}, so you can get a sense of what these look like.

\lstinputlisting[float=pht, frame=single, caption=An example Bib\LaTeX{} database, label=lst:thedatabase, language=TeX]{sources.bib}

Each entry in the database follows a standard format.
\begin{itemize}
\item The first line tells Bib\LaTeX{} the \emph{type} of the entry, using a code like \code{{@}Online} (indicating that the entry is of type \enquote{Online}).

\item
  The body of the entry appears between the curly braces \code{\{} and \code{\}}.

\item
  The first line of the body gives the entry a \emph{key}, which will be used to refer to the entry later.

\item
  Each line after that is of the form \enquote{\code{attribute = value,}}, and adds some piece of information to the entry.
  For example, you can tell Bib\LaTeX{} that Donald Knuth is the author of a text with the line
\begin{verbatim}
author = {Knuth, Donald},
\end{verbatim}
  Each value should be enclosed in curly braces, as in \cref{lst:thedatabase}.
\end{itemize}

\subsection{Structure of a database entry}
\label{s:structure}
For example, consider the entry for \code{knuth:aocp} in \cref{lst:thedatabase}.
Let's break it down line-by-line.

\begin{itemize}
\item
  The first line tells Bib\LaTeX{} that this entry represents a book.
\begin{verbatim}
@Book{
\end{verbatim}

\item
  The second line sets the citation key.
\begin{verbatim}
knuth:aocp,
\end{verbatim}

\item
  The third line sets the title of the book.
  Note that, as with all the other lines, the value is enclosed in curly braces.
\begin{verbatim}
title = {The Art of Computer Programming: Fundamental algorithms},
\end{verbatim}

\item
  The fourth line sets the author of the book.
  Bib\LaTeX{} is able to handle author fields very intelligently; it will take care of things like abbreviating and sorting names according to style.
  Thus, we include the full name.
  (Putting the last name first with a comma is good practice; it makes sure that Bib\LaTeX{} handles things correctly even when authors have multi-word last names.)
\begin{verbatim}
author = {Knuth, Donald Erwin},
\end{verbatim}

\item
  The remaining lines set publication data for the book.
\begin{verbatim}
series = {The Art of Computer Programming},
year = {1968},
isbn = {0-201-03801-3},
publisher = {Addison-Wesley},
}
\end{verbatim}
\end{itemize}

\subsection{Available entry types}
\label{s:entrytypes}
Bib\LaTeX{} supports a diverse array of entry types, supporting many different kinds of references.
A complete list is available in the documentation \cite{biblatex}.
A few of the most important ones for academic work are:
\begin{description}
\item[article]
  Used for articles which appear in journals or other publications but which have their own author and title.

\item[book]
  Used for a single-volume book.

\item[mvbook]
  Variant of \code{book}, used for books with multiple volumes.

\item[manual]
  Used for technical manuals and documentation.

\item[thesis]
  Used for theses and dissertations.

\item[unpublished]
  Used for material which has not been formally published in any venue.
  This is \emph{not} appropriate for preprints on a server like the arXiv.

\item[online]
  A catch-all for online materials.
  \emph{Do not use this unless no other type is a better fit!}
  For example, you should use \code{article} for a journal article which you downloaded from a website.
  However, this \emph{should} be used for preprints on a server like the arXiv, as shown in entry \code{vcuts} in \cref{lst:thedatabase}, which appears as \cite{vcuts}.
\end{description}

Each of these types has several \emph{required} fields (typically including \code{author} and \code{title}, but there may be more) and many \emph{optional} fields.
The Bib\LaTeX{} documentation is the definitive reference for this information.

\section{Editing the Bib\LaTeX{} database}
\label{s:editing}
Each of the Carleton templates comes with a Bib\LaTeX{} database example file called \code{sources.bib}.
To edit it in Write\LaTeX{}, just select \enquote{Project} in the upper left of the window, then click the edit icon next to this file.
Once you're done, go through the same process to get back to your main source file.

\section{Citations in your document}
\label{s:citing}
Once you've set up a Bib\LaTeX{} database entry, you need to cite it in your document.
To do this, use the \code{\textbackslash{}cite\{\}} command.
For example, to cite \textit{The Art of Computer Programming} using the entry from \cref{s:structure}, we just type \code{\textbackslash{}cite\{knuth:aocp\}}, which produces the following citation: \cite{knuth:aocp}.
In addition, this tells Bib\LaTeX{} to add this text to the bibliography; take a look at the References section at the end of the document to see the result.

If you prefer footnote citations, you can obtain them using the \code{\textbackslash{}footcite\{\}}, which has the same syntax as \code{\textbackslash{}cite\{\}}.
In this style, citing \code{\textbackslash{}footcite\{\}} results in this: \footcite{biblatex}, which looks a bit silly, but many styles define footnote citations that make sense.

A third option is the \code{\textbackslash{}autocite\{\}} command, which uses either \code{\textbackslash{}cite\{\}} or \code{\textbackslash{}footcite\{\}} depending on which is standard for the style.
Again, this uses the same syntax as \code{\textbackslash{}cite\{\}}.

\subsection{Citations with postnotes}
\label{s:postnotes}
Many academic citations include so-called \enquote{postnotes}, which give the reader additional information about where in a text they should look for a particular piece of information.
For example, if you want to read more about this topic, you should consult \cite[\S 3.7]{biblatex}.

To format such a citation, we only need to pass an optional argument to \code{\textbackslash{}cite\{\}} using square braces.
The previous citation was obtained by typing the following:
\begin{verbatim}
\cite[\S 3.7]{biblatex}
\end{verbatim}

\section{Bibliographic styles}
\label{s:styles}
Bib\LaTeX{} supports many different bibliographic styles, which control both the inline citations and the bibliography entries at the end of the document.
This document uses the \enquote{numeric} style, which produces inline citations like this one: \cite{biblatex}.
You can choose the style used by your document by changing the \code{style=?} option in the line
\begin{verbatim}
\usepackage[backend=bibtex,style=numeric]{biblatex}
\end{verbatim}
in the preamble.

A few of the major styles that Bib\LaTeX{} supports include:
\begin{description}
\item[mla]
  Formats citations according to the MLA style guide\footnote{This is surprisingly tricky---the older Bib\TeX{} system simply can't handle it!}.
  It would cite \cite{knuth:aocp} as \enquote{[Knu68]}.

\item[alphabetic]
  Uses alphabetic instead of numerical keys for inline citations.
  It would cite \cite{knuth:aocp} as \enquote{[Knu68]}.

\item[authoryear]
  Uses a more verbose author-year inline citation.
  It would cite \cite{knuth:aocp} as \enquote{(Knuth 1968)}.
\end{description}

\section{Obtaining Bib\LaTeX{} database entries}
\label{s:obtaining}
In practice, you won't need to do a lot of manual entry of Bib\LaTeX{} database entries.
For one thing, they're reusable---once you format an entry for a text once, you can use it over and over again throughout your work.
Just copy the entry from the old database and paste it into the new one.

In addition, Bib\TeX{} stanzas are available through many common systems for finding sources.
For example, if you look up a paper on \href{http://scholar.google.com}{Google Scholar} and click on the \enquote{Cite} link, the \enquote{Import into Bib\TeX{}} link will take you to a Bib\LaTeX{} database entry ready to copy-and-paste.
(Unfortunately, Google doesn't always do a great job with these, especially when the papers are preprints on the arXiv.
Still, it's a start!)
A similar feature is available through the AMS's MathSciNet reviews database.

\printbibliography
\end{document}
